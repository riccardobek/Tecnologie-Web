  \section{Accessibilit�}
Tutte le scelte progettuali sono state prese in base al tema dell'accesibilit�, seguendo le linee guida del W3C.\vspace{-0.3cm}
\subsection{Colori}
Si � evitato di utilizzare combinazioni di colori che potessero creare problemi di accessibilit�, usabilit� e comprensione del contenuto a persone affette da daltonismo. Si riportano di seguito degli screenshot di varie simulazioni di daltonismo della pagina Attivit�. Inoltre � stato mantenuto un contrasto pari a 7:1 tra il colore di sfondo e il colore dei testi.
\begin{figure}[h]
	\centering
	\footnotesize
	\stackunder[5pt]{\includegraphics[scale=0.25]{images/normal_color.png}}{Immagine Originale} \quad
	\stackunder[5pt]{\includegraphics[scale=0.25]{images/deut.png}}{Deutenatropia} \\
\end{figure}
\begin{figure}[h]
	\centering
	\footnotesize
	\stackunder[5pt]{\includegraphics[scale=0.25]{images/tren.png}}{Protanopia} \quad
	\stackunder[5pt]{\includegraphics[scale=0.25]{images/tritano.png}}{Tritanopia} \\
	\caption{Simulazione di daltonismo}
\end{figure} \\

\newpage
\noindent
Per migliorare il contrasto il colore dei link � stato modificato rendendo comunque riconoscibili i link visitati e non.

\subsection{Tag}
Per quanto concerne i tag che migliorano l'accessibilit�:
\begin{itemize}
 	\item Sono stati utilizzati i tag \texttt{alt} per le immagini seguendo le linee guida del W3C.
 	\item Le parole in lingua inglese sono state racchiuse in un tag \texttt{<span lang="en"> </span>} cos� da poter garantire una lettura corretta da parte degli screen reader.
 	\item Sono stati utilizzati i tag \texttt{tabindex} in modo da permettere la navigazione corretta del sito attraverso il tasto TAB. Inoltre, poich� l'intestazione del sito si ripete per ogni pagina, � stato introdotta una voce del men� non visibile \texttt{Salta intestazione} che permette agli utenti che utilizzano uno screen reader di saltare la lettura dell'intestazione e passare direttamente al contenuto.
 	\item \'{E} stato utilizzato il tag \texttt{scope} dove necessario seguendo i dettami del W3C. Inoltre � stato evitato l'utilizzo di tabelle per realizzare il layout del sito.
 	\item Per rendere accessibili i form, sono stati utilizzati i tag \texttt{label, fieldset e title}, assieme ad una gestione degli errori che comprende  controlli di validit� dell'input.
\end{itemize}
\hypertarget{aria}{\subsubsection{Tag WAI-ARIA}}
Sono stati utilizzati dei tag introdotti dalle specifiche WAI-ARIA che permettono di migliorare l'accessibilit� per gli utenti che utilizzano uno screen reader. 
\paragraph{div di errore per i form}\mbox{}\\
Per segnalare all'utente degli errori di input nei vari form � stato creato un 
\texttt{div} che ha i seguenti attributi:
\begin{itemize}
	\item \texttt{aria-live="assertive"} Le tecnologie assistive (AT), come uno screen reader, notificano immediatamente all'utente la regione dichiarata \texttt{assertive}, in questo caso il \texttt{div} di errore verr� notificato e verranno letti gli errori, permettendo all'utente di correggere l'input.  
	\item \texttt{aria-atomic="true"} Indica allo screen reader di leggere interamente la regione di errore e non solo i suoi cambiamenti. In questo modo se l'utente non ha corretto un errore segnalato in precedenza, questo viene notificato nuovamente.
\end{itemize}
Il \texttt{div} di errore viene nascosto, mostrato e modificato tramite JavaScript e se quest'ultimo dovesse essere disabilitato, allora all'invio dei dati, viene visualizzata una pagina di errore che elenca tutti gli eventuali errori di input.
\paragraph{div di dialog}\mbox{}\\
Per le finestre di dialogo create da noi (\emph{non quelle della libreria \texttt{jquery-confirm}}) sono stati utilizzati i seguenti tag:
\begin{itemize}
	\item \texttt{role="dialog"} Serve ad informare gli utenti che utilizzano uno screen reader che � apparsa una finestra di dialogo con la quale deve interagire.
	\item \texttt{aria-labelledby=}[id elemento che descrive la dialog] Permette allo screen reader di leggere il contenuto dell'elemento che ha l'\texttt{id} specificato nell'attributo \texttt{aria-labelledby}; nel nostro caso lo screen reader informer� l'utente quale finestra di dialogo � stata aperta leggendone il titolo.
\end{itemize}
Tutte le finestre di dialogo sono utilizzate nella parte interna del sito, ovvero nella pagina \texttt{Pannelo Utente} e nella pagina \texttt{Pannello Admin}.
\subsection{Dispositivi}
Nella progettazione si � tenuto conto del fatto che l'utenza avrebbe acceduto al sito da vari dispostivi. Alla luce di questa considerazione il sito adotta un layout resposivo grazie alla creazione di file \texttt{css} appositi (mobile e stampa). Il sito � stato testato su pi� dispositivi, browser e sistemi operativi possibili in modo tale da avere un feedback pi� completo possibile sulle scelte adottate (Vedere la sezione \hyperref[test]{Validazione e Test} per i dettagli).