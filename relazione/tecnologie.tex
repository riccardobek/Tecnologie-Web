\section{Tecnologie utilizzate}
Di seguito si descrivono le tecnologie e le motivazioni del loro utilizzio per la realizzazione del sito. Inoltre per ogni tecnologia si mettono in evidenza alcune meccaniche di implementazione che altrimenti potrebbero non essere ovvie guardando il codice.
\subsection{HTML 5}
La scelta di utilizzare \texttt{HTML 5}(rispetto a \texttt{XHTML strict}) � stata presa per garantire compatibilit� con il maggior numero di browser possibili. Per questo motivo, inoltre, � stato evitato l'utilizzo di tag  particolari (tag Meta o di markup strutturale).

\section{CSS}
Per la gestione del layout del sito, il \texttt{CSS embedded} � stato evitato completamente, a favore di una separazione completa tra layout e struttura attraverso l'utlizzo di fogli di stile: lo standard adottato � il \texttt{CSS 3.0}.
\\
In fase di realizzazione del sito, si � deciso di incorporare tutti i fogli di stile in un unico file(\texttt{deafult.css}), sono stati poi realizzati layout a parte per la versione mobile e per la stampa.
Sono poi presenti altri 2 fogli di stile: uno per il date-picker(vedi sezione 5.3.1) e un'altro necessario per il pdf di conferma prenotazione.

\subsection{PHP}
\subsubsection{Dinamicit�}
PHP svolge un ruolo fondamentale per il funzionamento del sito, infatti grazie a questa tecnologia abbiamo realizzato un sito dinamico i cui contenuti (le attivit� offerte) possono essere aggiunti, modificati ed eliminati attraverso il \hyperlink{pannelloAdmin}{pannello di amministrazione} che si trova nella parte interna del sito. Inoltre visto che sono presenti elementi che vengono ripetuti, come ad esempio l'intestazione e il menu, si � optato per una templetizzazione di essi e attraverso dei segnaposto (con la seguente sintassi: [\#SEGNAPOSTO]) che vengono rimpiazzati dal contenuto attraverso \texttt{PHP}. 

Le funzioni che gestiscono operazioni come il login, la prenotazione/cancellazione di attivit�, il caricamento delle immagini e la generazione di pdf sono collocate nella cartella \texttt{php}.
\subsubsection{Sicurezza}
Il tema della sicurezza ha avuto un ruolo fondamentale durante la realizzazione del sito, infatti sono state create delle funzioni di sicurezza che permettono di garantire il controllo completo sulle operazioni che vengono richieste al server. Tale controllo consiste nel verificare quale tipologia di utente ha fatto la richiesta attraverso le informazioni contenute nella sessione; se l'utente non � un amministratore allora il server rigetta la richiesta.
\\
Per le richieste HTTP, si � preferito un utilizzo del metodo \texttt{POST}, ideale per situazioni in cui bisogna trattare dati personali e sensibili degli utenti(come, ad esempio, la registrazione e il login); tutti i dati inviati al server vengono sanificati in modo tale da impedire attacchi SQL Injection, questo permette la 

\subsubsection{Caricamento delle immagini}
Nel momento in cui viene creata una nuova macroattivit�, vi � la possibilit� di caricare  2 immagini distinte.
Tale funzionalit� � stata implementata tramite una funzione PHP uploadImage, che posiziona l'immagine nelle cartella images/attivita/index e images/attivita/banner.
Una soluzione che risulta essere in linea con il principio della dinamicit� del sito gestita, ove possibile, da PHP.

\subsection{JavaScript}
\subsubsection{Librerie}
Di seguito sono elencate le librerie JavaScript utilizate ai fini del progetto.
\begin{itemize}
	\item \textbf{jquery-asDatepicker.js:} libreria che permette di visualizzare un date-picker per la selezione di una data della prenotazione. Per problemi dovuti all'accessibilit�, sono state necessarie alcune modifiche in modo da permettere a uno screen reader una corretta lettura del contenuto.
	 
	\item \textbf{jquery-confirm.js:} genera finestre di dialogo pop-up; anche in questo caso sono state necessarie alcune modifiche per rendere l'output accessibile a uno screen reader.  
\end{itemize}
\newpage
\subsection{Database}
Si � deciso di utilizzare PhpMyAdmin per la gestione del database contenente i dati del sito. 
Segue uno schema UML delle tabelle 
\begin{figure}[h]
	\centering
	\subfloat[\emph{Struttura del DB}]
	{\includegraphics[scale=0.7]{images/erdb.png}}
\end{figure}
\\
il database � inoltre privo di trigger, si � deciso infatti di gestire tramite \texttt{PHP} le varie operazioni riguardanti la gestione dei dati.
